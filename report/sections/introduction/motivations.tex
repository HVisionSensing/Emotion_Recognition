\phantomsection
\chapter{Motivations}

\noindent A facial expression is a visible manifestation of the effective state, cognitive activity, intent, personality, and psychopathology of a person \cite{DON99}; facial expressions play a significant role in human dialogue and interaction. Indeed, facial expressions carry more informations than mere speech, informations on which humans can relay for interaction. Facial expressions have a considerable effect on a listening interlocutor; a speaker facial expressions accounts for about 55 percent of the effect, 38 percent of the latter is conveyed by voice intonation and 7 percent by the spoken words \cite{PAN00}.
\newline

\noindent Since Antiquity, researchers have been interested in emotion and more particularly in emotion recognition. But one of the important studies on facial expression analysis impacting on the modern day science of automatic facial expression recognition was the work carried out by Charles Darwin \cite{BET12}. In 1872, Darwin wrote a treatise that established general expression principles and expression means for both humans and animals \cite{DAR04}. He also classified various kinds of expressions. This can be considered as the beginning of facial expression recognition.
\newline

\noindent Now, with the emergence of new technologies and computers, research is now focused on computer-based automatic facial expression recognition. Because facial expressions are major factors in human interaction, this research field will broaden the domain of Human-Machine Interaction. Indeed, emotion recognition will enable computers to be more responsive to users' emotions, and allow interactions to become more and more realistic. 
\newline

\noindent Another domain where facial expression recognition is an important issue is robotics. With the advances made in robotics, robots nowadays tend to mimic human emotion and react as as human-like as possible, especially for humanoid robots. However, since robots are being more and more present in our daily lives, they need to understand and recognize human emotions.
\newline

\noindent A lot of real time applications in the robotics field have already been created. For example, Bartlett et al. have successfully used their face expression recognition system to develop an animated character that mirrors the expressions of the user (called CU Animate) \cite{BAR03}. They have also been successful in deploying the recognition system on Sony's Aibo Robot and ATR's RoboVie \cite{BAR03}. Another interesting application has been demonstrated by Anderson and McOwen, called "EmotiChat" \cite{AND06}. It is a regular chatroom, except the fact that their facial expression recognition system is connected to the chat and convert the users' facial expressions into emoticons. Because facial expression recognition systems' robustness and reliability are constantly increasing, lots of innovative applications will appear.
\newline

\noindent There are also various other domains where emotion recognition can be used: Telecommunications, behavioural science, video games, animations, psychiatry, automobile safety, affect-sensitive music jukeboxes and televisions, educational software, etc \cite{BET12}.
\newline

\noindent Our project focuses on real-time facial expression recognition from a video stream. Indeed, facial expression recognition can be performed \textit{statically} on input images, or \textit{dynamically} on video sequences. Systems can also be \textit{obtrusive}, or \textit{non-obtrusive}, the former based on a device mounted on the user's head or body, therefore following each of his movements and perform facial expression recognition without much losses, while the latter can encounter difficulties if the user is not properly situated. However, non-obtrusive systems allow more natural user interactions. We chose our system to be non-obtrusive, and will detail further its setup in the next section.
\newline

\phantomsection
\section{Environment Setup}

\vspace{\baselineskip}
\noindent Our system will use the camera embedded into a Microsoft Kinect to record the user's video input, and we will consider a casual use of the camera, with the user sitting in front of the computer, the camera being next to it, as seen in \textbf{\color{red} Insert figure \& ref to figure}. This camera provides a 640$\times$480 pixels frame resolution, while recording at 30 FPS.
\newline

\noindent For development and training purposes we will use some pre-existing emotion datasets, in order to validate the efficiency of the system before testing it in real conditions.
\newline

\phantomsection
\section{Emotion Datasets}

\noindent Introduction bla bla bla
\newline

\subsection{JAFFE Database}

\vspace{\baselineskip}
\noindent bla bla bla
\newline

\subsection{KDFE Database}

\vspace{\baselineskip}
\noindent bla bla bla
\newline

\subsection{MSFDE Database}

\vspace{\baselineskip}
\noindent bla bla bla
\newline
