\phantomsection
\chapter{Motivations}
\label{ch:motivations}

\noindent A facial expression is a "visible manifestation of the effective state, cognitive activity, intent, personality, and psychopathology of a person" \cite{DON99}; facial expressions represent a huge part in dialogue and interaction with other humans. Indeed, facial expressions carry more informations than speech, informations on which humans can relay for interaction. Facial expressions have a considerable effect on a listening interlocutor \cite{PAN00}. 
\newline

\noindent Albert Mehrabian found that facial expressions represents $ 55\% $ of information received by a listener, while $ 38\% $ of information received comes form voice intonation and the remaining $ 7\% $ by the spoken words. To find this numbers, Albert Mehrabian conducted an experiment. This experiment was based only on one word \textit{maybe}. Three female subjects pronounced this word with different intonations to express different attitudes addressed to a inexistent listener. The different attitudes was, for example, \textit{like}, \textit{dislike} and \textit{neutral}. Each subject was recorded and taped. Then 17 female subjects were told to that first three female subjects were addressing someone and to guess their attitudes toward this someone. The second experiment was to shot the three female subjects while expressing the different attitudes. Then the 17 female subjects guessed again the attitudes of the three female subjects but this time they heard the recording at the same time that they saw the photographs. The output of these experiments was that the facial part was stronger than the vocal one by a ration of $ \frac{3}{2} $. It corresponds to the $ 55\% $ and the $ 38\% $ mentioned previously \cite{PAN00}. This percentage can be however applied for only certain specific situations. It is obvious that in some situations, the speech is more important than the facial component.
\newline

\noindent Since antiquity, researchers have been interested in emotion and more particularly in emotion recognition. One of the most important studies on facial expression analysis impacting on modern day science of automatic facial expression recognition is the work carried out by Charles Darwin \cite{BET12}. In 1872, Darwin wrote a book that established general expression principles, expression means and expression description for both humans and animals \cite{DAR04}. He also classified various kinds of expressions. This can be considered as the beginning of facial expression recognition.
\newline

\noindent Nowadays, with the emergence of new technologies and computers, research is now focused on computer-based automatic facial expression recognition. Because facial expressions are major factors in human interaction, this research field will improve the domain of Human-Machine Interaction. Indeed, emotion recognition will enable computers to be more responsive to users' emotions, and allow interactions to become more and more realistic. 
\newline

\noindent Another domain where facial expression recognition is an important issue is robotics. With the advances made in robotics, robots tend to mimic human emotion and react as as human-like as possible, especially for humanoid robots. Indeed, since robots are being more and more present in our daily lives, they need to understand and recognize human emotions.
\newline

\noindent A lot of applications in the robotics field have already been created. For example, Bartlett et al. have successfully used their face expression recognition system to develop an character that is animated and that mirrors the expressions of the user (called CU Animate) \cite{BAR03}. They have also been successful in deploying the recognition system on Sony's Aibo Robot and ATR's RoboVie \cite{BAR03}. Another interesting application has been demonstrated by Anderson and McOwen, called "EmotiChat" \cite{AND06}. It is a regular chatroom, except the fact that their facial expression recognition system is connected to the chat and convert the users' facial expressions into emoticons. Because facial expression recognition systems' robustness and reliability are constantly increasing, lots of innovative applications will appear.
\newline

\noindent There are also various other domains where emotion recognition can be used: Telecommunications, behavioural science, video games, animations, psychiatry, automobile safety, affect-sensitive music jukeboxes and televisions, educational software, etc \cite{BET12}.
\newline

\noindent This project focuses on facial expression recognition from a video stream. Indeed, facial expression recognition can be performed \textit{statically} on input images, or \textit{dynamically} on video sequences. Systems can also be \textit{obtrusive}, or \textit{non-obtrusive}, the former based on a device mounted on the user's head or body, therefore following each of his movements and perform facial expression recognition without much losses, while the latter can encounter difficulties if the user is not properly situated. However, non-obtrusive systems allow more natural user interactions. The system described in this report was chosen to be non-obtrusive, and its setup will be detailed further in the next section.
\newline