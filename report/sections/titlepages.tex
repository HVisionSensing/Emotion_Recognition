\pdfbookmark[0]{English title page}{label:titlepage_en}
\aautitlepage{%
  \englishprojectinfo{
    Project Title %title
  }{%
    Interactive Systems %theme
  }{%
    Fall Semester 2012 %project period
  }{%
    12gr942 % project group
  }{%
    %list of group members
    Maxime Coupez\\ 
    Kim-Adeline Miguel\\
    Julia Alexandra Vigo
  }{%
    %list of supervisors
    Zheng-Hua Tan\\
  }{%
    1 % number of printed copies
  }{%
    \today % date of completion
  }%
}{%department and address
  \textbf{Department of Electronic Systems}\\
  Fredrik Bajers Vej 7\\
  DK-9220 Aalborg Ø\\
  \href{http://es.aau.dk}{http://es.aau.dk}
}{% the abstract
Since the last decade, a lot of researches have been carried out about emotion recognition. The number of projects conducted in this field demonstrates the interest and the importance of systems which can recognize human mood.


In this project, an emotion recognition system is developed, using a Microsoft Kinect. This recognition is achieved in 3 steps: Face detection, extraction and classification of facial features, this structure being the usual modus operandi in emotion recognition research. 


Face detection is performed using Viola-Jones' algorithm, then Local Binary Patterns (LBP) are used to extract facial features. Finally, Support Vector Machines (SVM) classify these features into six predefined emotions.


The system is implemented to run on a computer using a Kinect and works for one person in front of it. The classifier is trained with the Cohn-Kanade database, which includes enough different faces to obtain a satisfying result.

}
